\documentclass{article}
\usepackage[top=3cm, bottom=3cm, outer=3cm, inner=3cm]{geometry}
\usepackage{multicol}
\usepackage{amsmath}
\usepackage{graphicx}
\usepackage{url}
\usepackage{hyperref}
\usepackage{listings}
\usepackage{array}
\usepackage{natbib}
\usepackage{multirow}
\usepackage[normalem]{ulem}
\useunder{\uline}{\ul}{}
\usepackage{xcolor}
\usepackage{colortbl}
\usepackage{caption}
\usepackage{float}

\newcolumntype{M}[1]{>{\centering\arraybackslash}m{#1}}
\newcolumntype{x}[1]{>{\centering\arraybackslash\hspace{0pt}}p{#1}}

%%%%%%%%%%%%%%%%%%%%%%%%%%%%%%%%%%%%%%%%%%%%%%%%%%%%%%%%%%%%%%%%%%%%%%%%%%%%
% INFORMACIÓN DEL DOCUMENTO
%%%%%%%%%%%%%%%%%%%%%%%%%%%%%%%%%%%%%%%%%%%%%%%%%%%%%%%%%%%%%%%%%%%%%%%%%%%%

\newcommand{\itemCourse}{Web Semántica y Ontologías}
\newcommand{\itemUniversity}{Universidad La Salle}
\newcommand{\itemFaculty}{Facultad de Ingeniería}
\newcommand{\itemSchool}{Carrera Profesional de Ingeniería de Software}
\newcommand{\itemAcademic}{2024 - II}
\newcommand{\itemDate}{Diciembre 2024}
\newcommand{\itemVersion}{1.0}
\newcommand{\itemProfesor}{Ing. Marco Antonio Camacho Alatrista}
\newcommand{\itemProjectName}{SmartCompareMarket}
\newcommand{\itemProjectNumber}{Proyecto 16 - Nivel 2}

%%%%%%%%%%%%%%%%%%%%%%%%%%%%%%%%%%%%%%%%%%%%%%%%%%%%%%%%%%%%%%%%%%%%%%%%%%%%
% CONFIGURACIÓN DE IDIOMA
%%%%%%%%%%%%%%%%%%%%%%%%%%%%%%%%%%%%%%%%%%%%%%%%%%%%%%%%%%%%%%%%%%%%%%%%%%%%

\usepackage[english,spanish]{babel}
\usepackage[utf8]{inputenc}
\AtBeginDocument{\selectlanguage{spanish}}
\renewcommand{\figurename}{Figura}
\renewcommand{\refname}{Referencias}
\renewcommand{\tablename}{Tabla}
\renewcommand{\contentsname}{Índice de Contenidos}

%%%%%%%%%%%%%%%%%%%%%%%%%%%%%%%%%%%%%%%%%%%%%%%%%%%%%%%%%%%%%%%%%%%%%%%%%%%%
% ESTILO DE PÁGINA
%%%%%%%%%%%%%%%%%%%%%%%%%%%%%%%%%%%%%%%%%%%%%%%%%%%%%%%%%%%%%%%%%%%%%%%%%%%%

\usepackage{fancyhdr}
\pagestyle{fancy}
\fancyhf{}
\setlength{\headheight}{40pt}
\renewcommand{\headrulewidth}{1pt}
\renewcommand{\footrulewidth}{1pt}

\fancyhead[L]{\raisebox{-0.2\height}{\includegraphics[width=3cm]{logo_salle.png}}}
\fancyhead[C]{\fontsize{7}{7}\selectfont	\itemUniversity \\ \itemFaculty \\ \itemSchool \\ \textbf{\itemCourse}}
\fancyfoot[C]{Página \thepage}
\fancyfoot[R]{\itemProjectName}

%%%%%%%%%%%%%%%%%%%%%%%%%%%%%%%%%%%%%%%%%%%%%%%%%%%%%%%%%%%%%%%%%%%%%%%%%%%%
% ESTILO DE CÓDIGO
%%%%%%%%%%%%%%%%%%%%%%%%%%%%%%%%%%%%%%%%%%%%%%%%%%%%%%%%%%%%%%%%%%%%%%%%%%%%

\definecolor{dkgreen}{rgb}{0,0.6,0}
\definecolor{gray}{rgb}{0.5,0.5,0.5}
\definecolor{mauve}{rgb}{0.58,0,0.82}
\definecolor{codebackground}{rgb}{0.95, 0.95, 0.92}
\definecolor{tablebackground}{rgb}{0.0, 0.45, 0.63}
\definecolor{successgreen}{rgb}{0.0, 0.5, 0.0}
\definecolor{warningorange}{rgb}{0.85, 0.55, 0.0}

\lstset{
	frame=tb,
	language=bash,
	aboveskip=3mm,
	belowskip=3mm,
	showstringspaces=false,
	columns=flexible,
	basicstyle={\small\ttfamily},
	numbers=none,
	numberstyle=\tiny\color{gray},
	keywordstyle=\color{blue},
	commentstyle=\color{dkgreen},
	stringstyle=\color{mauve},
	breaklines=true,
	breakatwhitespace=true,
	tabsize=3,
	backgroundcolor=\color{codebackground},
	literate={á}{{\'a}}1 {é}{{\'e}}1 {í}{{\'i}}1 {ó}{{\'o}}1 {ú}{{\'u}}1 {ñ}{{\~n}}1,
}

%%%%%%%%%%%%%%%%%%%%%%%%%%%%%%%%%%%%%%%%%%%%%%%%%%%%%%%%%%%%%%%%%%%%%%%%%%%%
% INICIO DEL DOCUMENTO
%%%%%%%%%%%%%%%%%%%%%%%%%%%%%%%%%%%%%%%%%%%%%%%%%%%%%%%%%%%%%%%%%%%%%%%%%%%%

\begin{document}

%%%%%%%%%%%%%%%%%%%%%%%%%%%%%%%%%%%%%%%%%%%%%%%%%%%%%%%%%%%%%%%%%%%%%%%%%%%%
% PORTADA
%%%%%%%%%%%%%%%%%%%%%%%%%%%%%%%%%%%%%%%%%%%%%%%%%%%%%%%%%%%%%%%%%%%%%%%%%%%%

\begin{titlepage}
	\centering
	\vspace*{1cm}
	
	\includegraphics[width=4cm]{logo_salle.png}
	\vspace{1cm}
	
	{\Huge\bfseries MANUAL DE USUARIO\par}
	\vspace{0.5cm}
	{\LARGE\bfseries \itemProjectName\par}
	\vspace{0.3cm}
	{\Large Marketplace Semántico con Comparación Inteligente\par}
	\vspace{1cm}
	
	{\Large\textbf{\itemProjectNumber}\par}
	\vspace{0.3cm}
	{\large Proyecto Final - \itemCourse\par}
	\vspace{0.5cm}
	{\large Versión \itemVersion\par}
	\vspace{1.5cm}
	
	{\large\itemUniversity\par}
	{\large\itemFaculty\par}
	{\large\itemSchool\par}
	\vspace{1cm}
	
	\begin{table}[H]
		\centering
		\begin{tabular}{|x{4cm}|x{10cm}|}
			\hline 
			\rowcolor{tablebackground}
			\color{white}\textbf{Docente} & \color{white}\textbf{\itemProfesor} \\
			\hline 
		\end{tabular}
	\end{table}
	
	\begin{table}[H]
		\centering
		\begin{tabular}{|x{7cm}|x{7cm}|}
			\hline 
			\rowcolor{tablebackground}
			\color{white}\textbf{Estudiante} & \color{white}\textbf{Correo Electrónico} \\
			\hline 
			Álvaro André Machaca Meléndez & amachacam@ulasalle.edu.pe \\
			\hline
		\end{tabular}
	\end{table}
	
	\vspace{1cm}
	{\large\itemDate\par}
	{\large Ciclo Académico: \itemAcademic\par}
	
\end{titlepage}

%%%%%%%%%%%%%%%%%%%%%%%%%%%%%%%%%%%%%%%%%%%%%%%%%%%%%%%%%%%%%%%%%%%%%%%%%%%%
% TABLA DE CONTENIDOS
%%%%%%%%%%%%%%%%%%%%%%%%%%%%%%%%%%%%%%%%%%%%%%%%%%%%%%%%%%%%%%%%%%%%%%%%%%%%

\tableofcontents
\newpage

%%%%%%%%%%%%%%%%%%%%%%%%%%%%%%%%%%%%%%%%%%%%%%%%%%%%%%%%%%%%%%%%%%%%%%%%%%%%
% CONTENIDO DEL MANUAL
%%%%%%%%%%%%%%%%%%%%%%%%%%%%%%%%%%%%%%%%%%%%%%%%%%%%%%%%%%%%%%%%%%%%%%%%%%%%

\section{Introducción}

\subsection{¿Qué es SmartCompareMarket?}

\textbf{SmartCompareMarket} es una plataforma web inteligente para comparar y recibir recomendaciones de productos electrónicos (Laptops, Smartphones, Tablets). A diferencia de los comparadores tradicionales, este sistema utiliza tecnologías de \textbf{Web Semántica} e \textbf{Inteligencia Artificial} para:

\begin{itemize}
	\item \textbf{Entender} las características de los productos mediante ontologías OWL 2
	\item \textbf{Clasificar automáticamente} productos usando reglas SWRL (ej: detectar si una laptop es ``Gamer'')
	\item \textbf{Comparar inteligentemente} productos con un motor de scoring multi-factor
	\item \textbf{Recomendar} productos personalizados según las preferencias del usuario
	\item \textbf{Validar} que las especificaciones de productos sean consistentes
	\item \textbf{Buscar} productos usando consultas SPARQL semánticas
\end{itemize}

\subsection{Características Principales}

\begin{table}[H]
	\centering
	\begin{tabular}{|c|p{10cm}|}
		\hline
		\rowcolor{tablebackground}
		\color{white}\textbf{Característica} & \color{white}\textbf{Descripción} \\
		\hline
		🧠 Razonamiento Automático & Clasifica productos automáticamente (LaptopGamer si RAM $\geq$ 16GB) \\
		\hline
		🔍 Comparación Inteligente & Motor que calcula scores basados en 9 factores y reglas SWRL \\
		\hline
		📊 Consultas SPARQL & Búsqueda semántica con filtros avanzados \\
		\hline
		🎯 Recomendaciones & Sistema basado en perfil de usuario y razonamiento \\
		\hline
		✅ Validación & Detecta errores e inconsistencias en especificaciones \\
		\hline
		🔗 Relaciones & Detecta compatibilidad, incompatibilidad y equivalencias \\
		\hline
	\end{tabular}
	\caption{Características principales del sistema}
\end{table}

\subsection{¿Para quién es este manual?}

Este manual está diseñado para \textbf{usuarios finales} que desean:

\begin{itemize}
	\item Instalar y ejecutar el sistema en su computadora
	\item Usar la plataforma para comparar productos
	\item Obtener recomendaciones personalizadas
\end{itemize}

\textbf{Nota:} No se requieren conocimientos técnicos avanzados. Las instrucciones están escritas paso a paso.

\newpage

%%%%%%%%%%%%%%%%%%%%%%%%%%%%%%%%%%%%%%%%%%%%%%%%%%%%%%%%%%%%%%%%%%%%%%%%%%%%

\section{Requisitos del Sistema}

\subsection{Requisitos de Hardware}

\begin{table}[H]
	\centering
	\begin{tabular}{|l|l|l|}
		\hline
		\rowcolor{tablebackground}
		\color{white}\textbf{Componente} & \color{white}\textbf{Mínimo} & \color{white}\textbf{Recomendado} \\
		\hline
		Procesador & Dual Core 2.0 GHz & Quad Core 2.5+ GHz \\
		\hline
		Memoria RAM & 4 GB & 8 GB o más \\
		\hline
		Espacio en Disco & 2 GB libres & 5 GB libres \\
		\hline
		Conexión Internet & Requerida (instalación) & Requerida \\
		\hline
	\end{tabular}
	\caption{Requisitos de hardware del sistema}
\end{table}

\subsection{Requisitos de Software}

Antes de comenzar, debes tener instalados los siguientes programas:

\begin{table}[H]
	\centering
	\begin{tabular}{|l|l|l|}
		\hline
		\rowcolor{tablebackground}
		\color{white}\textbf{Software} & \color{white}\textbf{Versión} & \color{white}\textbf{Comando de Verificación} \\
		\hline
		Python & 3.11 o superior & \texttt{python --version} \\
		\hline
		Node.js & 18.0 o superior & \texttt{node --version} \\
		\hline
		npm & 9.0 o superior & \texttt{npm --version} \\
		\hline
		Java JDK & 11 o superior & \texttt{java -version} \\
		\hline
		Git (opcional) & Cualquiera & \texttt{git --version} \\
		\hline
	\end{tabular}
	\caption{Software necesario y comandos de verificación}
\end{table}

\textbf{⚠️ IMPORTANTE:} Java es necesario para el razonador semántico Pellet.

\subsection{Navegadores Compatibles}

\begin{table}[H]
	\centering
	\begin{tabular}{|l|l|l|}
		\hline
		\rowcolor{tablebackground}
		\color{white}\textbf{Navegador} & \color{white}\textbf{Versión} & \color{white}\textbf{Estado} \\
		\hline
		Google Chrome & 90+ & ✅ Recomendado \\
		\hline
		Mozilla Firefox & 88+ & ✅ Compatible \\
		\hline
		Microsoft Edge & 90+ & ✅ Compatible \\
		\hline
		Safari & 14+ & ✅ Compatible \\
		\hline
	\end{tabular}
	\caption{Navegadores compatibles}
\end{table}

\newpage

%%%%%%%%%%%%%%%%%%%%%%%%%%%%%%%%%%%%%%%%%%%%%%%%%%%%%%%%%%%%%%%%%%%%%%%%%%%%

\section{Instalación Paso a Paso}

\subsection{Paso 1: Verificar Prerrequisitos}

Abre una \textbf{terminal/consola de comandos} y ejecuta estos comandos uno por uno:

\begin{lstlisting}
python --version
\end{lstlisting}

\begin{figure}[H]
    \centering
    \includegraphics[width=0.9\textwidth]{logo_salle.png} 
    \caption{Terminal mostrando Python 3.11.x o superior}
    \label{fig:python_version}
\end{figure}

\begin{lstlisting}
node --version
\end{lstlisting}

\begin{figure}[H]
    \centering
    \includegraphics[width=0.9\textwidth]{logo_salle.png} 
    \caption{Terminal mostrando v18.x.x o superior}
    \label{fig:node_version}
\end{figure}

\begin{lstlisting}
npm --version
\end{lstlisting}

\begin{figure}[H]
    \centering
    \includegraphics[width=0.9\textwidth]{logo_salle.png} 
    \caption{Terminal mostrando 9.x.x o superior}
    \label{fig:npm_version}
\end{figure}

\begin{lstlisting}
java -version
\end{lstlisting}

\begin{figure}[H]
    \centering
    \includegraphics[width=0.9\textwidth]{logo_salle.png} 
    \caption{Terminal mostrando openjdk version 11.x.x o similar}
    \label{fig:java_version}
\end{figure}

\subsection{Paso 2: Descargar el Proyecto}

\subsubsection{Opción A: Usando Git (Recomendado)}

\begin{lstlisting}
git clone https://github.com/AlvaroMachaca0503/Web_Semantica.git
cd Web_Semantica
\end{lstlisting}

\begin{figure}[H]
    \centering
    \includegraphics[width=0.9\textwidth]{logo_salle.png} 
    \caption{Terminal mostrando el proceso de clonación del repositorio}
    \label{fig:git_clone}
\end{figure}

\subsubsection{Opción B: Descarga Manual}

\begin{enumerate}
	\item Ve a la página del repositorio en GitHub
	\item Haz clic en el botón verde \textbf{``Code''}
	\item Selecciona \textbf{``Download ZIP''}
	\item Extrae el archivo ZIP en una carpeta de tu elección
	\item Abre una terminal en esa carpeta
\end{enumerate}

\begin{figure}[H]
    \centering
    \includegraphics[width=0.9\textwidth]{logo_salle.png} 
    \caption{Carpeta del proyecto mostrando las subcarpetas backend y frontend}
    \label{fig:folder_structure}
\end{figure}

\subsection{Paso 3: Instalar Dependencias del Backend}

\begin{enumerate}
	\item Navega a la carpeta del backend:
	
\begin{lstlisting}
cd backend
\end{lstlisting}

	\item Crea un entorno virtual de Python:
	
	\textbf{En Windows:}
\begin{lstlisting}
python -m venv venv
venv\Scripts\activate
\end{lstlisting}

	\textbf{En Mac/Linux:}
\begin{lstlisting}
python3 -m venv venv
source venv/bin/activate
\end{lstlisting}
\end{enumerate}

\begin{figure}[H]
    \centering
    \includegraphics[width=0.9\textwidth]{logo_salle.png} 
    \caption{Terminal mostrando (venv) al inicio}
    \label{fig:venv_active}
\end{figure}

\begin{enumerate}
	\setcounter{enumi}{2}
	\item Instala las dependencias:
	
\begin{lstlisting}
pip install -r requirements.txt
\end{lstlisting}
\end{enumerate}

\begin{figure}[H]
    \centering
    \includegraphics[width=0.9\textwidth]{logo_salle.png} 
    \caption{Terminal mostrando dependencias instaladas exitosamente}
    \label{fig:pip_install}
\end{figure}

\subsection{Paso 4: Instalar Dependencias del Frontend}

\begin{enumerate}
	\item Abre \textbf{otra terminal} (mantén la del backend abierta)
	\item Navega a la carpeta del frontend:
	
\begin{lstlisting}
cd frontend
\end{lstlisting}

	\item Instala las dependencias:
	
\begin{lstlisting}
npm install
\end{lstlisting}
\end{enumerate}

\textbf{Nota:} Este proceso puede tardar 2-5 minutos dependiendo de tu conexión.

\begin{figure}[H]
    \centering
    \includegraphics[width=0.9\textwidth]{logo_salle.png} 
    \caption{Terminal mostrando paquetes de Node.js instalados}
    \label{fig:npm_install}
\end{figure}

\newpage

%%%%%%%%%%%%%%%%%%%%%%%%%%%%%%%%%%%%%%%%%%%%%%%%%%%%%%%%%%%%%%%%%%%%%%%%%%%%

\section{Ejecución del Proyecto}

\subsection{Iniciar el Backend (Servidor API)}

\begin{enumerate}
	\item Asegúrate de estar en la carpeta \texttt{backend} con el entorno virtual activado
	\item Ejecuta:
	
\begin{lstlisting}
python main.py
\end{lstlisting}

	\item \textbf{Espera} hasta ver estos mensajes de éxito:
\end{enumerate}

\begin{lstlisting}
[OK] Ontologia cargada: 60+ productos
[OK] Razonador Pellet ejecutado exitosamente
[OK] Reglas SWRL aplicadas
INFO:     Uvicorn running on http://0.0.0.0:5000
\end{lstlisting}

\begin{figure}[H]
    \centering
    \includegraphics[width=0.9\textwidth]{logo_salle.png} 
    \caption{Terminal mostrando inicio exitoso del backend}
    \label{fig:backend_start}
\end{figure}

\textbf{⚠️ ¡NO CIERRES ESTA TERMINAL!} El servidor debe permanecer ejecutándose.

\subsection{Iniciar el Frontend (Interfaz Web)}

\begin{enumerate}
	\item En la \textbf{segunda terminal}, asegúrate de estar en la carpeta \texttt{frontend}
	\item Ejecuta:
	
\begin{lstlisting}
npm run dev
\end{lstlisting}

	\item Espera hasta ver:
\end{enumerate}

\begin{lstlisting}
  VITE v5.x.x  ready in XXX ms

  ->  Local:   http://localhost:5173/
  ->  Network: use --host to expose
\end{lstlisting}

\begin{figure}[H]
    \centering
    \includegraphics[width=0.9\textwidth]{logo_salle.png} 
    \caption{Terminal mostrando Vite ejecutándose}
    \label{fig:frontend_vite}
\end{figure}

\subsection{Abrir la Aplicación}

\begin{enumerate}
	\item Abre tu \textbf{navegador web} (Chrome recomendado)
	\item Escribe en la barra de direcciones: \texttt{http://localhost:5173}
	\item Presiona \textbf{Enter}
\end{enumerate}

\begin{figure}[H]
    \centering
    \includegraphics[width=0.9\textwidth]{logo_salle.png} 
    \caption{Página principal del sistema mostrando el catálogo de productos}
    \label{fig:home_page}
\end{figure}

\subsection{URLs Importantes}

\begin{table}[H]
	\centering
	\begin{tabular}{|l|l|}
		\hline
		\rowcolor{tablebackground}
		\color{white}\textbf{Servicio} & \color{white}\textbf{URL} \\
		\hline
		Frontend (Interfaz) & http://localhost:5173 \\
		\hline
		Backend (API) & http://localhost:5000 \\
		\hline
		Documentación API (Swagger) & http://localhost:5000/docs \\
		\hline
	\end{tabular}
	\caption{URLs principales del sistema}
\end{table}

\newpage

%%%%%%%%%%%%%%%%%%%%%%%%%%%%%%%%%%%%%%%%%%%%%%%%%%%%%%%%%%%%%%%%%%%%%%%%%%%%

\section{Guía de Uso del Sistema}

\subsection{Navegación Principal}

La aplicación tiene \textbf{tres secciones principales} accesibles desde el menú superior:

\begin{table}[H]
	\centering
	\begin{tabular}{|c|c|p{8cm}|}
		\hline
		\rowcolor{tablebackground}
		\color{white}\textbf{Sección} & \color{white}\textbf{Icono} & \color{white}\textbf{Descripción} \\
		\hline
		Inicio & 🏠 & Catálogo completo de productos \\
		\hline
		Comparar & ⚖️ & Comparación inteligente de productos \\
		\hline
		Recomendaciones & 💡 & Recomendaciones personalizadas \\
		\hline
	\end{tabular}
	\caption{Secciones principales de navegación}
\end{table}

\begin{figure}[H]
    \centering
    \includegraphics[width=0.9\textwidth]{logo_salle.png} 
    \caption{Menú de navegación superior mostrando las tres opciones}
    \label{fig:nav_menu}
\end{figure}

\subsection{Explorar el Catálogo de Productos}

En la \textbf{página de Inicio}, verás tarjetas de productos. Cada tarjeta muestra:

\begin{itemize}
	\item 📷 Imagen del producto
	\item 🏷️ Nombre y marca
	\item 💵 Precio (con descuento si aplica)
	\item ⭐ Calificación de usuarios
	\item 🖥️ Especificaciones técnicas (RAM, Almacenamiento, etc.)
	\item 🎮 Badges especiales (ej: ``Laptop Gamer'')
\end{itemize}

\begin{figure}[H]
    \centering
    \includegraphics[width=0.9\textwidth]{logo_salle.png} 
    \caption{Tarjeta de producto mostrando nombre, precio, especificaciones y badge}
    \label{fig:product_card}
\end{figure}

\subsubsection{Usando los Filtros de Búsqueda}

\begin{enumerate}
	\item Usa la \textbf{barra de búsqueda} para buscar por nombre
	\item Puedes seleccionar entre \textbf{2 y 5 productos}
	\item Verás una \textbf{barra flotante} en la parte inferior indicando cuántos has seleccionado
\end{enumerate}

\begin{figure}[H]
    \centering
    \includegraphics[width=0.9\textwidth]{logo_salle.png} 
    \caption{Tres tarjetas de productos con una seleccionada}
    \label{fig:selected_cards}
\end{figure}

\begin{figure}[H]
    \centering
    \includegraphics[width=0.9\textwidth]{logo_salle.png} 
    \caption{Barra flotante inferior de comparación}
    \label{fig:compare_bar}
\end{figure}

\subsubsection{Paso 2: Ir a la página de comparación}

Haz clic en el botón \textbf{``Comparar''} de la barra flotante.

\begin{figure}[H]
    \centering
    \includegraphics[width=0.9\textwidth]{logo_salle.png} 
    \caption{Página de comparación mostrando la tabla comparativa}
    \label{fig:compare_page}
\end{figure}

\subsubsection{Paso 3: Analizar los resultados}

La página de comparación muestra:

\textbf{A. Ganador Global (parte superior)}
\begin{itemize}
	\item El sistema determina automáticamente el \textbf{mejor producto}
	\item Muestra el \textbf{score numérico} (0-100 puntos)
	\item Explica la \textbf{razón} de por qué ganó
\end{itemize}

\begin{figure}[H]
    \centering
    \includegraphics[width=0.9\textwidth]{logo_salle.png} 
    \caption{Sección del Ganador con nombre, score y razón}
    \label{fig:winner_section}
\end{figure}

\textbf{B. Tabla Comparativa (centro)}
\begin{itemize}
	\item Cada columna es un producto
	\item Cada fila es una característica (Precio, RAM, Batería, etc.)
	\item Los valores \textbf{en verde} son los mejores de cada fila
	\item Los valores \textbf{en amarillo} indican empate
\end{itemize}

\begin{figure}[H]
    \centering
    \includegraphics[width=0.9\textwidth]{logo_salle.png} 
    \caption{Tabla comparativa con celdas resaltadas}
    \label{fig:compare_table}
\end{figure}

\textbf{C. Inferencias SWRL (parte inferior)}
\begin{itemize}
	\item Muestra las \textbf{reglas inteligentes} aplicadas
	\item Ejemplos: ``🎮 LaptopGamer detectado'', ``💰 Es mejor opción que ProductoX''
\end{itemize}

\begin{figure}[H]
    \centering
    \includegraphics[width=0.9\textwidth]{logo_salle.png} 
    \caption{Sección Reglas SWRL Aplicadas mostrando inferencias}
    \label{fig:swrl_inferences}
\end{figure}

\newpage

\subsection{Obtener Recomendaciones Personalizadas}

\subsubsection{Paso 1: Ir a Recomendaciones}

Haz clic en \textbf{``Recomendaciones''} en el menú superior.

\subsubsection{Paso 2: Configurar preferencias}

En el \textbf{panel izquierdo}, ajusta:

\begin{table}[H]
	\centering
	\begin{tabular}{|l|p{6cm}|l|}
		\hline
		\rowcolor{tablebackground}
		\color{white}\textbf{Preferencia} & \color{white}\textbf{Descripción} & \color{white}\textbf{Ejemplo} \\
		\hline
		💵 Presupuesto máximo & Cuánto puedes gastar & \$1500 \\
		\hline
		📁 Categoría preferida & Tipo de producto & Laptop \\
		\hline
		🧠 RAM mínima & Memoria RAM mínima & 16 GB \\
		\hline
		💾 Almacenamiento mínimo & Disco duro mínimo & 512 GB \\
		\hline
		⭐ Calificación mínima & Puntuación de usuarios & 4.0 \\
		\hline
	\end{tabular}
	\caption{Opciones de preferencias del usuario}
\end{table}

\begin{figure}[H]
    \centering
    \includegraphics[width=0.9\textwidth]{logo_salle.png} 
    \caption{Panel de preferencias con sliders y selectores configurados}
    \label{fig:preferences_panel}
\end{figure}

\subsubsection{Paso 3: Ver las recomendaciones}

El \textbf{panel derecho} mostrará:

\begin{enumerate}
	\item Lista de productos ordenados por \textbf{relevancia}
	\item Para cada producto:
	\begin{itemize}
		\item \textbf{Score de match} (0-100\%)
		\item \textbf{Razón} de la recomendación
		\item Especificaciones principales
	\end{itemize}
\end{enumerate}

\begin{figure}[H]
    \centering
    \includegraphics[width=0.9\textwidth]{logo_salle.png} 
    \caption{Lista de recomendaciones con scores y razones}
    \label{fig:recommendations_list}
\end{figure}

\newpage

%%%%%%%%%%%%%%%%%%%%%%%%%%%%%%%%%%%%%%%%%%%%%%%%%%%%%%%%%%%%%%%%%%%%%%%%%%%%

\section{Sistema de Comparación Inteligente}

\subsection{Algoritmo de Scoring}

El motor de comparación evalúa productos usando \textbf{9 factores ponderados}:

\begin{table}[H]
	\centering
	\begin{tabular}{|l|c|l|}
		\hline
		\rowcolor{tablebackground}
		\color{white}\textbf{Factor} & \color{white}\textbf{Peso} & \color{white}\textbf{Criterio} \\
		\hline
		🔋 Batería & 20\% & Mayor es mejor \\
		\hline
		⭐ Calificación & 18\% & Mayor es mejor \\
		\hline
		💵 Precio & 14\% & \textbf{Menor es mejor} \\
		\hline
		📺 Resolución & 10\% & Mayor es mejor \\
		\hline
		🧠 RAM & 10\% & Mayor es mejor \\
		\hline
		💾 Almacenamiento & 10\% & Mayor es mejor \\
		\hline
		🛡️ Garantía & 7\% & Mayor es mejor \\
		\hline
		📐 Pantalla & 6\% & Mayor es mejor \\
		\hline
		⚖️ Peso & 5\% & \textbf{Menor es mejor} \\
		\hline
	\end{tabular}
	\caption{Factores de scoring y sus pesos}
\end{table}

\textbf{Bonus por reglas SWRL:}
\begin{itemize}
	\item +2 puntos si el producto ``es mejor opción que'' otro
	\item +10 puntos si es detectado como ``Laptop Gamer''
\end{itemize}

\subsection{Clasificación Automática (SWRL)}

El sistema clasifica productos automáticamente usando reglas inteligentes:

\begin{table}[H]
	\centering
	\begin{tabular}{|l|l|l|}
		\hline
		\rowcolor{tablebackground}
		\color{white}\textbf{Regla} & \color{white}\textbf{Condición} & \color{white}\textbf{Clasificación} \\
		\hline
		DetectarGamer & Laptop con RAM $\geq$ 16GB & → LaptopGamer 🎮 \\
		\hline
		EncontrarMejorPrecio & Mismo producto, menor precio & → esMejorOpcionQue \\
		\hline
		ClasificarPositivas & Reseña con calificación $\geq$ 4 & → Reseña\_Positiva \\
		\hline
		ClasificarNegativas & Reseña con calificación $\leq$ 2 & → Reseña\_Negativa \\
		\hline
	\end{tabular}
	\caption{Reglas SWRL implementadas}
\end{table}

\begin{figure}[H]
    \centering
    \includegraphics[width=0.9\textwidth]{logo_salle.png} 
    \caption{Producto mostrando el badge Laptop Gamer clasificado automáticamente}
    \label{fig:gamer_badge}
\end{figure}

\newpage

%%%%%%%%%%%%%%%%%%%%%%%%%%%%%%%%%%%%%%%%%%%%%%%%%%%%%%%%%%%%%%%%%%%%%%%%%%%%

\section{Solución de Problemas Comunes}

\subsection{Error: ``python no se reconoce como comando''}

\textbf{Problema:} Python no está instalado o no está en el PATH.

\textbf{Solución:}
\begin{enumerate}
	\item Descarga Python desde: \url{https://www.python.org/downloads/}
	\item Durante la instalación, \textbf{marca la casilla ``Add Python to PATH''}
	\item Reinicia la terminal
\end{enumerate}

\begin{figure}[H]
    \centering
    \includegraphics[width=0.9\textwidth]{logo_salle.png} 
    \caption{Instalador de Python con casilla Add Python to PATH marcada}
    \label{fig:python_installer}
\end{figure}

\subsection{Error: ``npm no se reconoce como comando''}

\textbf{Problema:} Node.js no está instalado.

\textbf{Solución:}
\begin{enumerate}
	\item Descarga Node.js desde: \url{https://nodejs.org/}
	\item Elige la versión \textbf{LTS} (recomendada)
	\item Instala y reinicia la terminal
\end{enumerate}

\subsection{Error: ``No se puede conectar al servidor'' en el frontend}

\textbf{Problema:} El backend no está ejecutándose.

\textbf{Solución:}
\begin{enumerate}
	\item Verifica que la terminal del backend muestre:
	\begin{lstlisting}
INFO: Uvicorn running on http://0.0.0.0:5000
	\end{lstlisting}
	\item Si no, vuelve a ejecutar \texttt{python main.py}
\end{enumerate}

\begin{figure}[H]
    \centering
    \includegraphics[width=0.9\textwidth]{logo_salle.png} 
    \caption{Terminal del backend ejecutándose correctamente}
    \label{fig:backend_running}
\end{figure}

\subsection{Error: ``Error loading ontology''}

\textbf{Problema:} Java no está instalado (necesario para Pellet).

\end{enumerate}

\begin{figure}[H]
    \centering
    \includegraphics[width=0.9\textwidth]{logo_salle.png} 
    \caption{Navegador mostrando JSON de productos}
    \label{fig:json_browser}
\end{figure}

\newpage

%%%%%%%%%%%%%%%%%%%%%%%%%%%%%%%%%%%%%%%%%%%%%%%%%%%%%%%%%%%%%%%%%%%%%%%%%%%%

\section{Preguntas Frecuentes (FAQ)}

\subsection{¿Qué significa ``Inferencia SWRL''?}

Son \textbf{reglas lógicas programadas} en la ontología. Por ejemplo: \textit{``Si una Laptop tiene RAM $\geq$ 16GB, entonces se clasifica como LaptopGamer''}. El sistema aplica estas reglas automáticamente.

\subsection{¿Cómo se decide el ``Ganador'' en una comparación?}

El sistema calcula un \textbf{score de 0 a 100} considerando:
\begin{enumerate}
	\item Los 9 factores técnicos (ver Tabla 6)
	\item Bonus por reglas SWRL aplicadas
	\item Relación calidad-precio
\end{enumerate}

El producto con el score más alto gana.

\subsection{¿Por qué algunos productos tienen el badge ``Laptop Gamer''?}

El sistema detectó automáticamente que tienen \textbf{RAM de 16GB o más}, lo cual es típico de laptops para gaming según las reglas SWRL.

\subsection{¿Necesito Java si el sistema es Python?}

Sí. El razonador \textbf{Pellet} que ejecuta las reglas SWRL está escrito en Java. La librería Owlready2 lo necesita para el razonamiento semántico.

\subsection{¿Necesito Internet para usar el sistema?}

\begin{itemize}
	\item \textbf{Para instalar:} Sí, para descargar dependencias
	\item \textbf{Para usar:} No, funciona localmente una vez instalado
\end{itemize}

\newpage

%%%%%%%%%%%%%%%%%%%%%%%%%%%%%%%%%%%%%%%%%%%%%%%%%%%%%%%%%%%%%%%%%%%%%%%%%%%%

\section{Lista de Capturas de Pantalla}

Para completar este manual, se deben incluir las siguientes capturas de pantalla:

\begin{table}[H]
	\centering
	\begin{tabular}{|c|p{11cm}|}
		\hline
		\rowcolor{tablebackground}
		\color{white}\textbf{\#} & \color{white}\textbf{Descripción de la Captura} \\
		\hline
		1 & Terminal mostrando \texttt{python --version} \\
		\hline
		2 & Terminal mostrando \texttt{node --version} \\
		\hline
		3 & Terminal mostrando \texttt{npm --version} \\
		\hline
		4 & Terminal mostrando \texttt{java -version} \\
		\hline
		5 & Terminal mostrando proceso de clonación del repositorio \\
		\hline
		6 & Carpeta del proyecto con subcarpetas backend y frontend \\
		\hline
		7 & Terminal con \texttt{(venv)} activado \\
		\hline
		8 & Instalación pip completada exitosamente \\
		\hline
		9 & npm install completado \\
		\hline
		10 & Backend ejecutándose con Pellet OK \\
		\hline
		11 & Frontend con Vite ejecutándose \\
		\hline
		12 & Página principal del sistema \\
		\hline
		13 & Menú de navegación \\
		\hline
		14 & Tarjeta de producto con badge Laptop Gamer \\
		\hline
		15 & Panel de filtros de búsqueda \\
		\hline
		16 & Productos seleccionados para comparar \\
		\hline
		17 & Barra flotante de comparación \\
		\hline
		18 & Página de comparación completa \\
		\hline
		19 & Sección del ganador \\
		\hline
		20 & Tabla comparativa con valores resaltados \\
		\hline
		21 & Sección de reglas SWRL aplicadas \\
		\hline
		22 & Panel de preferencias de recomendaciones \\
		\hline
		23 & Lista de recomendaciones con scores \\
		\hline
		24 & Producto con badge Laptop Gamer \\
		\hline
		25 & Instalador Python con Add to PATH \\
		\hline
		26 & Backend ejecutándose correctamente \\
		\hline
		27 & JSON de productos en navegador \\
		\hline
	\end{tabular}
	\caption{Lista de capturas de pantalla requeridas}
\end{table}

\textbf{Total de capturas requeridas: 27}

\newpage

%%%%%%%%%%%%%%%%%%%%%%%%%%%%%%%%%%%%%%%%%%%%%%%%%%%%%%%%%%%%%%%%%%%%%%%%%%%%

\section{Conclusiones}

Este manual ha cubierto todos los aspectos necesarios para instalar, configurar, ejecutar y utilizar \textbf{SmartCompareMarket}. Los principales logros del proyecto incluyen:

\begin{itemize}
	\item \textbf{Ontología OWL 2 compleja:} 48 clases, 30+ propiedades, 60+ individuos
	\item \textbf{Reglas SWRL funcionales:} Clasificación automática de productos
	\item \textbf{Motor de comparación inteligente:} Scoring multi-factor + inferencias
	\item \textbf{Búsqueda semántica:} Consultas SPARQL con filtros avanzados
	\item \textbf{Recomendaciones personalizadas:} Basadas en perfil y razonamiento
	\item \textbf{Validación de consistencia:} Detección de errores en especificaciones
	\item \textbf{Frontend moderno:} React + TypeScript + Tailwind CSS
	\item \textbf{Backend robusto:} FastAPI + Pellet + Owlready2
\end{itemize}

\subsection{Tecnologías Utilizadas}

\begin{table}[H]
	\centering
	\begin{tabular}{|l|l|}
		\hline
		\rowcolor{tablebackground}
		\color{white}\textbf{Componente} & \color{white}\textbf{Tecnología} \\
		\hline
		Backend & Python 3.11 + FastAPI + Uvicorn \\
		\hline
		Ontología & OWL 2 + SWRL + Pellet \\
		\hline
		Consultas & SPARQL + RDFlib \\
		\hline
		Frontend & React 18 + TypeScript + Vite \\
		\hline
		Estilos & Tailwind CSS + Shadcn/UI \\
		\hline
	\end{tabular}
	\caption{Stack tecnológico del proyecto}
\end{table}

\vspace{1cm}

\begin{center}
	\textit{Gracias por utilizar SmartCompareMarket.}
	
	\textit{¡Éxito en tu experiencia de comparación inteligente!}
\end{center}

\newpage

%%%%%%%%%%%%%%%%%%%%%%%%%%%%%%%%%%%%%%%%%%%%%%%%%%%%%%%%%%%%%%%%%%%%%%%%%%%%

\section{Información del Proyecto}

\begin{table}[H]
	\centering
	\begin{tabular}{|p{5cm}|p{9cm}|}
		\hline
		\rowcolor{tablebackground}
		\color{white}\textbf{Campo} & \color{white}\textbf{Información} \\
		\hline
		Universidad & \itemUniversity \\
		\hline
		Facultad & \itemFaculty \\
		\hline
		Carrera & \itemSchool \\
		\hline
		Curso & \itemCourse \\
		\hline
		Docente & \itemProfesor \\
		\hline
		Ciclo Académico & \itemAcademic \\
		\hline
		Fecha & \itemDate \\
		\hline
		Proyecto & \itemProjectNumber \\
		\hline
		Nombre del Sistema & \itemProjectName \\
		\hline
		Versión del Manual & \itemVersion \\
		\hline
	\end{tabular}
	\caption{Información académica del proyecto}
\end{table}

\subsection{Autor}

\begin{table}[H]
	\centering
	\begin{tabular}{|l|l|}
		\hline
		\rowcolor{tablebackground}
		\color{white}\textbf{Nombre} & \color{white}\textbf{Correo Electrónico} \\
		\hline
		Álvaro André Machaca Meléndez & amachacam@ulasalle.edu.pe \\
		\hline
	\end{tabular}
	\caption{Autor del proyecto}
\end{table}

\subsection{Repositorio}

\begin{itemize}
	\item \textbf{GitHub:} \url{https://github.com/AlvaroMachaca0503/Web_Semantica}
\end{itemize}

\subsection{Historial de Versiones}

\begin{table}[H]
	\centering
	\begin{tabular}{|c|c|p{8cm}|}
		\hline
		\rowcolor{tablebackground}
		\color{white}\textbf{Versión} & \color{white}\textbf{Fecha} & \color{white}\textbf{Cambios} \\
		\hline
		1.0 & Diciembre 2024 & Versión inicial del manual de usuario \\
		\hline
	\end{tabular}
	\caption{Control de versiones del manual}
\end{table}

\end{document}