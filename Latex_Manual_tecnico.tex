\documentclass{article}
\usepackage[top=3cm, bottom=3cm, outer=3cm, inner=3cm]{geometry}
\usepackage{multicol}
\usepackage{amsmath}
\usepackage{graphicx}
\usepackage{url}
\usepackage{hyperref}
\usepackage{listings}
\usepackage{array}
\usepackage{natbib}
\usepackage{multirow}
\usepackage[normalem]{ulem}
\useunder{\uline}{\ul}{}
\usepackage{xcolor}
\usepackage{colortbl}
\usepackage{caption}
\usepackage{float}
\usepackage{longtable}

\newcolumntype{M}[1]{>{\centering\arraybackslash}m{#1}}
\newcolumntype{x}[1]{>{\centering\arraybackslash\hspace{0pt}}p{#1}}

%%%%%%%%%%%%%%%%%%%%%%%%%%%%%%%%%%%%%%%%%%%%%%%%%%%%%%%%%%%%%%%%%%%%%%%%%%%%
% INFORMACIÓN DEL DOCUMENTO
%%%%%%%%%%%%%%%%%%%%%%%%%%%%%%%%%%%%%%%%%%%%%%%%%%%%%%%%%%%%%%%%%%%%%%%%%%%%

\newcommand{\itemCourse}{Web Semántica y Ontologías}
\newcommand{\itemUniversity}{Universidad La Salle}
\newcommand{\itemFaculty}{Facultad de Ingeniería}
\newcommand{\itemSchool}{Carrera Profesional de Ingeniería de Software}
\newcommand{\itemAcademic}{2024 - II}
\newcommand{\itemDate}{Diciembre 2024}
\newcommand{\itemVersion}{1.0}
\newcommand{\itemProfesor}{Ing. Marco Antonio Camacho Alatrista}
\newcommand{\itemProjectName}{SmartCompareMarket}
\newcommand{\itemProjectNumber}{Proyecto 16 - Nivel 2}

%%%%%%%%%%%%%%%%%%%%%%%%%%%%%%%%%%%%%%%%%%%%%%%%%%%%%%%%%%%%%%%%%%%%%%%%%%%%
% CONFIGURACIÓN DE IDIOMA
%%%%%%%%%%%%%%%%%%%%%%%%%%%%%%%%%%%%%%%%%%%%%%%%%%%%%%%%%%%%%%%%%%%%%%%%%%%%

\usepackage[english,spanish]{babel}
\usepackage[utf8]{inputenc}
\AtBeginDocument{\selectlanguage{spanish}}
\renewcommand{\figurename}{Figura}
\renewcommand{\refname}{Referencias}
\renewcommand{\tablename}{Tabla}
\renewcommand{\contentsname}{Índice de Contenidos}

%%%%%%%%%%%%%%%%%%%%%%%%%%%%%%%%%%%%%%%%%%%%%%%%%%%%%%%%%%%%%%%%%%%%%%%%%%%%
% ESTILO DE PÁGINA
%%%%%%%%%%%%%%%%%%%%%%%%%%%%%%%%%%%%%%%%%%%%%%%%%%%%%%%%%%%%%%%%%%%%%%%%%%%%

\usepackage{fancyhdr}
\pagestyle{fancy}
\fancyhf{}
\setlength{\headheight}{40pt}
\renewcommand{\headrulewidth}{1pt}
\renewcommand{\footrulewidth}{1pt}

\fancyhead[L]{\raisebox{-0.2\height}{\includegraphics[width=3cm]{logo_salle.png}}}
\fancyhead[C]{\fontsize{7}{7}\selectfont	\itemUniversity \\ \itemFaculty \\ \itemSchool \\ \textbf{\itemCourse}}
\fancyfoot[C]{Página \thepage}
\fancyfoot[R]{\itemProjectName}

%%%%%%%%%%%%%%%%%%%%%%%%%%%%%%%%%%%%%%%%%%%%%%%%%%%%%%%%%%%%%%%%%%%%%%%%%%%%
% ESTILO DE CÓDIGO
%%%%%%%%%%%%%%%%%%%%%%%%%%%%%%%%%%%%%%%%%%%%%%%%%%%%%%%%%%%%%%%%%%%%%%%%%%%%

\definecolor{dkgreen}{rgb}{0,0.6,0}
\definecolor{gray}{rgb}{0.5,0.5,0.5}
\definecolor{mauve}{rgb}{0.58,0,0.82}
\definecolor{codebackground}{rgb}{0.95, 0.95, 0.92}
\definecolor{tablebackground}{rgb}{0.0, 0.45, 0.63}
\definecolor{successgreen}{rgb}{0.0, 0.5, 0.0}
\definecolor{warningorange}{rgb}{0.85, 0.55, 0.0}

\lstset{
	frame=tb,
	language=bash,
	aboveskip=3mm,
	belowskip=3mm,
	showstringspaces=false,
	columns=flexible,
	basicstyle={\small\ttfamily},
	numbers=none,
	numberstyle=\tiny\color{gray},
	keywordstyle=\color{blue},
	commentstyle=\color{dkgreen},
	stringstyle=\color{mauve},
	breaklines=true,
	breakatwhitespace=true,
	tabsize=3,
	backgroundcolor=\color{codebackground},
	literate={á}{{\'a}}1 {é}{{\'e}}1 {í}{{\'i}}1 {ó}{{\'o}}1 {ú}{{\'u}}1 {ñ}{{\~n}}1,
}

\lstdefinestyle{python}{
	language=Python,
	keywordstyle=\color{blue},
	stringstyle=\color{mauve},
	commentstyle=\color{dkgreen},
}

\lstdefinestyle{sparql}{
	language=SQL,
	keywordstyle=\color{blue},
	stringstyle=\color{mauve},
	morekeywords={PREFIX, SELECT, WHERE, FILTER, OPTIONAL},
}

\lstdefinestyle{json}{
	basicstyle={\small\ttfamily},
	stringstyle=\color{dkgreen},
}

%%%%%%%%%%%%%%%%%%%%%%%%%%%%%%%%%%%%%%%%%%%%%%%%%%%%%%%%%%%%%%%%%%%%%%%%%%%%
% INICIO DEL DOCUMENTO
%%%%%%%%%%%%%%%%%%%%%%%%%%%%%%%%%%%%%%%%%%%%%%%%%%%%%%%%%%%%%%%%%%%%%%%%%%%%

\begin{document}

%%%%%%%%%%%%%%%%%%%%%%%%%%%%%%%%%%%%%%%%%%%%%%%%%%%%%%%%%%%%%%%%%%%%%%%%%%%%
% PORTADA
%%%%%%%%%%%%%%%%%%%%%%%%%%%%%%%%%%%%%%%%%%%%%%%%%%%%%%%%%%%%%%%%%%%%%%%%%%%%

\begin{titlepage}
	\centering
	\vspace*{1cm}
	
	\includegraphics[width=4cm]{logo_salle.png}
	\vspace{1cm}
	
	{\Huge\bfseries MANUAL TÉCNICO\par}
	\vspace{0.5cm}
	{\LARGE\bfseries \itemProjectName\par}
	\vspace{0.3cm}
	{\Large Marketplace Semántico con Comparación Inteligente\par}
	\vspace{1cm}
	
	{\Large\textbf{\itemProjectNumber}\par}
	\vspace{0.3cm}
	{\large Proyecto Final - \itemCourse\par}
	\vspace{0.5cm}
	{\large Versión \itemVersion\par}
	\vspace{1.5cm}
	
	{\large\itemUniversity\par}
	{\large\itemFaculty\par}
	{\large\itemSchool\par}
	\vspace{1cm}
	
	\begin{table}[H]
		\centering
		\begin{tabular}{|x{4cm}|x{10cm}|}
			\hline 
			\rowcolor{tablebackground}
			\color{white}\textbf{Docente} & \color{white}\textbf{\itemProfesor} \\
			\hline 
		\end{tabular}
	\end{table}
	
	\begin{table}[H]
		\centering
		\begin{tabular}{|x{7cm}|x{7cm}|}
			\hline 
			\rowcolor{tablebackground}
			\color{white}\textbf{Estudiante} & \color{white}\textbf{Correo Electrónico} \\
			\hline 
			Álvaro André Machaca Meléndez & amachacam@ulasalle.edu.pe \\
			\hline
		\end{tabular}
	\end{table}
	
	\vspace{1cm}
	{\large\itemDate\par}
	{\large Ciclo Académico: \itemAcademic\par}
	
\end{titlepage}

%%%%%%%%%%%%%%%%%%%%%%%%%%%%%%%%%%%%%%%%%%%%%%%%%%%%%%%%%%%%%%%%%%%%%%%%%%%%
% TABLA DE CONTENIDOS
%%%%%%%%%%%%%%%%%%%%%%%%%%%%%%%%%%%%%%%%%%%%%%%%%%%%%%%%%%%%%%%%%%%%%%%%%%%%

\tableofcontents
\newpage

%%%%%%%%%%%%%%%%%%%%%%%%%%%%%%%%%%%%%%%%%%%%%%%%%%%%%%%%%%%%%%%%%%%%%%%%%%%%
% CONTENIDO DEL MANUAL TÉCNICO
%%%%%%%%%%%%%%%%%%%%%%%%%%%%%%%%%%%%%%%%%%%%%%%%%%%%%%%%%%%%%%%%%%%%%%%%%%%%

\section{Descripción General del Sistema}

\subsection{Propósito}

\textbf{SmartCompareMarket} es un sistema de marketplace que implementa tecnologías de \textbf{Web Semántica} para proporcionar:

\begin{itemize}
	\item Comparación inteligente de productos electrónicos
	\item Clasificación automática mediante razonamiento OWL
	\item Recomendaciones personalizadas basadas en ontologías
	\item Búsquedas semánticas con SPARQL
	\item Validación de consistencia de datos
\end{itemize}

\subsection{Alcance Técnico}

El sistema implementa los siguientes requisitos funcionales del \textbf{Proyecto 16 - Nivel 2}:

\begin{table}[H]
	\centering
	\begin{tabular}{|c|p{8cm}|c|}
		\hline
		\rowcolor{tablebackground}
		\color{white}\textbf{Req.} & \color{white}\textbf{Descripción} & \color{white}\textbf{Estado} \\
		\hline
		RF1 & Ontología de productos con jerarquías complejas & [OK] 100\% \\
		\hline
		RF2 & Modelado de equivalencias semánticas & [PENDING] 70\% \\
		\hline
		RF3 & Reglas de inferencia para compatibilidades & [OK] 100\% \\
		\hline
		RF4 & Motor de comparación con razonamiento & [OK] 100\% \\
		\hline
		RF5 & Búsqueda con filtros semánticos (SPARQL) & [OK] 100\% \\
		\hline
		RF6 & Recomendaciones basadas en perfil de usuario & [OK] 100\% \\
		\hline
		RF7 & Consultas SPARQL para análisis de mercado & [PENDING] 60\% \\
		\hline
		RF8 & Clasificación automática (subsunción OWL) & [PENDING] 75\% \\
		\hline
		RF9 & Validación de consistencia de especificaciones & [OK] 100\% \\
		\hline
	\end{tabular}
	\caption{Cumplimiento de requisitos funcionales}
\end{table}

\textbf{Total: 8.5 de 9 requisitos completamente funcionales (94\%)}

\newpage

%%%%%%%%%%%%%%%%%%%%%%%%%%%%%%%%%%%%%%%%%%%%%%%%%%%%%%%%%%%%%%%%%%%%%%%%%%%%

\section{Arquitectura del Sistema}

\subsection{Diagrama de Arquitectura}

El sistema implementa una arquitectura de \textbf{5 capas}:

\begin{verbatim}
+---------------------------------------------------------------+
|                    CAPA DE PRESENTACIÓN                        |
|  Frontend: React 18 + TypeScript + Vite + Tailwind CSS        |
|  Páginas: HomePage, ComparePage, RecommendationsPage          |
+----------------------------+----------------------------------+
                             | HTTP/REST (JSON) Puerto 5173->5000
+----------------------------v----------------------------------+
|                        CAPA DE API                             |
|  Backend: FastAPI + Uvicorn + Pydantic                        |
|  Routers: products, compare, recommendations, search, swrl    |
+----------------------------+----------------------------------+
                             |
+----------------------------v----------------------------------+
|                    CAPA DE SERVICIOS                           |
|  ProductService | ComparisonService | RecommendationService   |
|  ValidationService | SPARQLQueries                            |
+----------------------------+----------------------------------+
                             |
+----------------------------v----------------------------------+
|                  CAPA DE RAZONAMIENTO                          |
|  InferenceEngine | SWRLEngine | OntologyLoader                |
|  Owlready2 + RDFlib + Pellet                                  |
+----------------------------+----------------------------------+
                             |
+----------------------------v----------------------------------+
|                     CAPA DE DATOS                              |
|  Ontología OWL 2: SmartCompareMarket.owl (2900+ líneas)       |
|  48 clases | 30+ propiedades | 60+ individuos | 4 reglas SWRL |
+---------------------------------------------------------------+
\end{verbatim}

\begin{figure}[H]
    \centering
    \includegraphics[width=0.9\textwidth]{logo_salle.png} 
    \caption{Diagrama de arquitectura del sistema}
    \label{fig:arch_diagram}
\end{figure}

\subsection{Flujo de Datos}

\begin{verbatim}
[Usuario] -> [React Frontend] -> [HTTP Request] -> [FastAPI Router]
                                                         |
                                                [Service Layer]
                                                         |
                                  [InferenceEngine / SPARQLQueries]
                                                         |
                                    [Owlready2 + Pellet Reasoner]
                                                         |
                                       [SmartCompareMarket.owl]
                                                         |
                                                [JSON Response]
                                                         |
                                           [React State Update]
                                                         |
                                                 [UI Renderizado]
\end{verbatim}

\newpage

%%%%%%%%%%%%%%%%%%%%%%%%%%%%%%%%%%%%%%%%%%%%%%%%%%%%%%%%%%%%%%%%%%%%%%%%%%%%

\section{Tecnologías Utilizadas}

\subsection{Stack del Backend}

\begin{table}[H]
	\centering
	\begin{tabular}{|l|l|p{6cm}|}
		\hline
		\rowcolor{tablebackground}
		\color{white}\textbf{Tecnología} & \color{white}\textbf{Versión} & \color{white}\textbf{Propósito} \\
		\hline
		Python & 3.11+ & Lenguaje principal \\
		\hline
		FastAPI & 0.109.0 & Framework web REST \\
		\hline
		Uvicorn & 0.27.0 & Servidor ASGI \\
		\hline
		Pydantic & 2.6.0 & Validación de datos \\
		\hline
		Owlready2 & 0.46 & Manipulación de ontologías OWL \\
		\hline
		RDFlib & 7.0.0 & Motor SPARQL \\
		\hline
		Pellet & 2.4+ & Razonador OWL+SWRL (requiere Java) \\
		\hline
	\end{tabular}
	\caption{Stack tecnológico del backend}
\end{table}

\begin{figure}[H]
    \centering
    \includegraphics[width=0.9\textwidth]{logo_salle.png} 
    \caption{Archivo requirements.txt mostrando las dependencias}
    \label{fig:requirements}
\end{figure}

\subsection{Stack del Frontend}

\begin{table}[H]
	\centering
	\begin{tabular}{|l|l|p{6cm}|}
		\hline
		\rowcolor{tablebackground}
		\color{white}\textbf{Tecnología} & \color{white}\textbf{Versión} & \color{white}\textbf{Propósito} \\
		\hline
		React & 18.3.1 & Framework UI \\
		\hline
		TypeScript & 5.8.3 & Tipado estático \\
		\hline
		Vite & 5.4.19 & Build tool y dev server \\
		\hline
		Tailwind CSS & 3.4.17 & Framework CSS \\
		\hline
		Shadcn/UI & latest & Componentes UI \\
		\hline
		React Router & 6.30.1 & Enrutamiento SPA \\
		\hline
		Axios & 1.13.2 & Cliente HTTP \\
		\hline
		Lucide React & 0.462.0 & Iconografía \\
		\hline
	\end{tabular}
	\caption{Stack tecnológico del frontend}
\end{table}

\begin{figure}[H]
    \centering
    \includegraphics[width=0.9\textwidth]{logo_salle.png} 
    \caption{Archivo package.json mostrando las dependencias}
    \label{fig:packagejson}
\end{figure}

\subsection{Tecnologías Semánticas}

\begin{table}[H]
	\centering
	\begin{tabular}{|l|p{10cm}|}
		\hline
		\rowcolor{tablebackground}
		\color{white}\textbf{Tecnología} & \color{white}\textbf{Descripción} \\
		\hline
		OWL 2 & Web Ontology Language 2 para modelar conocimiento \\
		\hline
		SWRL & Semantic Web Rule Language para reglas de inferencia \\
		\hline
		SPARQL & Lenguaje de consultas para RDF/OWL \\
		\hline
		Pellet & Razonador que soporta OWL 2 + SWRL \\
		\hline
	\end{tabular}
	\caption{Tecnologías de Web Semántica}
\end{table}

\newpage

%%%%%%%%%%%%%%%%%%%%%%%%%%%%%%%%%%%%%%%%%%%%%%%%%%%%%%%%%%%%%%%%%%%%%%%%%%%%

\section{Estructura del Proyecto}

\subsection{Estructura de Directorios}

\begin{verbatim}
WebsemanticaProyect/
|-- backend/                          # Servidor API
|   |-- api/                          # Configuracion de la API
|   |-- data/                         # Datos de configuracion
|   |   |-- comparison_weights.json   # Pesos para scoring
|   |-- models/                       # Modelos Pydantic
|   |   |-- product.py                # Modelo de Producto
|   |   |-- comparison.py             # Modelo de Comparacion
|   |   |-- recommendation.py         # Modelo de Recomendacion
|   |-- ontology/                     # Archivos de ontologia
|   |   |-- SmartCompareMarket.owl    # ONTOLOGIA PRINCIPAL (2900+ lineas)
|   |   |-- owl_helpers.py            # Utilidades para OWL
|   |-- reasoning/                    # Capa de razonamiento
|   |   |-- inference_engine.py       # Motor de inferencias
|   |   |-- swrl_engine.py            # Motor de reglas SWRL
|   |   |-- ontology_loader.py        # Cargador de ontologia
|   |-- routers/                      # Endpoints de la API
|   |   |-- products.py               # /api/v1/products
|   |   |-- compare.py                # /api/v1/compare
|   |   |-- recommendations.py        # /api/v1/recommendations
|   |   |-- search.py                 # /api/v1/search
|   |   |-- swrl.py                   # /api/v1/swrl
|   |   |-- validation.py             # /api/v1/validate
|   |-- services/                     # Logica de negocio
|   |   |-- product_service.py        
|   |   |-- comparison_service.py     # Motor de comparacion (445 lineas)
|   |   |-- recommendation_service.py # Sistema de recomendaciones
|   |   |-- validation_service.py     # Validacion de datos
|   |-- sparql/                       # Consultas SPARQL
|   |   |-- queries.py                # Consultas predefinidas
|   |-- main.py                       # PUNTO DE ENTRADA
|   |-- requirements.txt              # Dependencias Python
|
|-- frontend/                         # Aplicacion React
|   |-- src/
|   |   |-- components/               # Componentes reutilizables
|   |   |-- pages/                    # Paginas principales
|   |   |   |-- HomePage.tsx          # Catalogo de productos
|   |   |   |-- ComparePage.tsx       # Comparacion inteligente
|   |   |   |-- RecommendationsPage.tsx # Recomendaciones
|   |   |-- services/                 # Servicios HTTP
|   |   |-- App.tsx                   # Componente raiz
|   |-- package.json                  # Dependencias Node.js
|
|-- README.md                         # Documentacion principal
|-- MANUAL_USUARIO.md                 # Manual de usuario
|-- MANUAL_TECNICO.md                 # Manual tecnico
\end{verbatim}

\begin{figure}[H]
    \centering
    \includegraphics[width=0.9\textwidth]{logo_salle.png} 
    \caption{Estructura de carpetas del proyecto en explorador de archivos}
    \label{fig:folder_structure}
\end{figure}

\newpage

%%%%%%%%%%%%%%%%%%%%%%%%%%%%%%%%%%%%%%%%%%%%%%%%%%%%%%%%%%%%%%%%%%%%%%%%%%%%

\section{Ontología OWL 2}

\subsection{Características Generales}

\begin{table}[H]
	\centering
	\begin{tabular}{|l|l|}
		\hline
		\rowcolor{tablebackground}
		\color{white}\textbf{Propiedad} & \color{white}\textbf{Valor} \\
		\hline
		Archivo & \texttt{backend/ontology/SmartCompareMarket.owl} \\
		\hline
		Tamaño & $\sim$2,900 líneas \\
		\hline
		Namespace & \texttt{http://smartcompare.com/ontologia\#} \\
		\hline
		Formato & RDF/XML \\
		\hline
	\end{tabular}
	\caption{Características de la ontología}
\end{table}

\subsection{Jerarquía de Clases}

\begin{verbatim}
owl:Thing
|-- Producto (raiz de productos)
|   |-- Electronica
|   |   |-- Smartphone
|   |   |-- Tablet
|   |   |-- Computadora
|   |       |-- Desktop
|   |       |-- Laptop
|   |           |-- LaptopGamer (inferida por SWRL)
|   |-- Moda
|   |   |-- Ropa
|   |   |-- Calzado
|   |-- Hogar
|       |-- Muebles
|       |-- Electrodomesticos
|
|-- Usuario
|   |-- Cliente
|   |-- Vendedor
|
|-- Resena
    |-- Resena_Positiva (inferida por SWRL)
    |-- Resena_Negativa (inferida por SWRL)
\end{verbatim}

\begin{figure}[H]
    \centering
    \includegraphics[width=0.9\textwidth]{logo_salle.png} 
    \caption{Jerarquía de clases en Protégé, tab Classes}
    \label{fig:protege_classes}
\end{figure}

\newpage

\subsection{Propiedades de Datos (Data Properties)}

\begin{table}[H]
	\centering
	\begin{tabular}{|l|l|l|p{4cm}|}
		\hline
		\rowcolor{tablebackground}
		\color{white}\textbf{Propiedad} & \color{white}\textbf{Dominio} & \color{white}\textbf{Rango} & \color{white}\textbf{Descripción} \\
		\hline
		tienePrecio & Producto & xsd:float & Precio en USD \\
		\hline
		tieneDescuento & Producto & xsd:float & Porcentaje de descuento \\
		\hline
		tieneRAM\_GB & Electronica & xsd:integer & RAM en GB \\
		\hline
		tieneAlmacenamiento\_GB & Electronica & xsd:integer & Almacenamiento en GB \\
		\hline
		tienePulgadas & Electronica & xsd:float & Tamaño de pantalla \\
		\hline
		resolucionPantalla & Electronica & xsd:string & Resolución (ej: ``1920x1080'') \\
		\hline
		bateriaCapacidad\_mAh & Electronica & xsd:integer & Capacidad de batería \\
		\hline
		procesadorModelo & Electronica & xsd:string & Modelo del procesador \\
		\hline
		procesadorVelocidad\_GHz & Electronica & xsd:float & Velocidad del CPU \\
		\hline
		numeroNucleosCPU & Electronica & xsd:integer & Número de núcleos \\
		\hline
		tarjetaGrafica & Computadora & xsd:string & GPU \\
		\hline
		garantiaMeses & Producto & xsd:integer & Garantía en meses \\
		\hline
		esEquivalenteTecnico & Symmetric & Productos técnicamente equivalentes \\
		\hline
		esMejorOpcionQue & Transitive & Producto A es mejor opción que B \\
		\hline
		tieneMejorRAMQue & - & Comparación de RAM \\
		\hline
	\end{tabular}
	\caption{Propiedades de objeto de la ontología}
\end{table}
\begin{itemize}
	\item \texttt{iPhone15\_Barato} (RAM: 6GB, 128GB, Precio: \$799)
	\item \texttt{Samsung\_Galaxy\_S24} (RAM: 8GB, 256GB, Precio: \$899)
	\item \texttt{Pixel\_8\_Pro} (RAM: 12GB, 256GB, Precio: \$999)
\end{itemize}

\textbf{Tablets:}
\begin{itemize}
	\item \texttt{iPad\_Pro\_12} (RAM: 16GB, 512GB, Precio: \$1099)
	\item \texttt{Samsung\_Tab\_S9} (RAM: 12GB, 256GB, Precio: \$849)
\end{itemize}

\begin{figure}[H]
    \centering
    \includegraphics[width=0.9\textwidth]{logo_salle.png} 
    \caption{Individuos en Protégé, tab Individuals}
    \label{fig:protege_individuals}
\end{figure}

\newpage

%%%%%%%%%%%%%%%%%%%%%%%%%%%%%%%%%%%%%%%%%%%%%%%%%%%%%%%%%%%%%%%%%%%%%%%%%%%%

\section{Reglas SWRL}

\subsection{Ubicación en la Ontología}

Las reglas SWRL se encuentran en las líneas \textbf{1964-2300} del archivo \texttt{SmartCompareMarket.owl}.

\subsection{Reglas Implementadas}

\subsubsection{Regla 1: DetectarGamer}

\begin{lstlisting}[style=sparql]
Laptop(?l) AND tieneRAM_GB(?l, ?ram) AND greaterThanOrEqual(?ram, 16)
  -> LaptopGamer(?l)
\end{lstlisting}

\textbf{Explicación técnica:}
\begin{itemize}
	\item \textbf{Antecedente:} Un individuo \texttt{?l} es de clase \texttt{Laptop} Y tiene propiedad \texttt{tieneRAM\_GB} con valor \texttt{?ram} Y ese valor es $\geq$ 16
	\item \textbf{Consecuente:} El individuo \texttt{?l} se clasifica como miembro de la clase \texttt{LaptopGamer}
	\item \textbf{Efecto:} Subsunción automática por el razonador Pellet
\end{itemize}

\begin{figure}[H]
    \centering
    \includegraphics[width=0.9\textwidth]{logo_salle.png} 
    \caption{Regla DetectarGamer en Protégé, tab SWRL}
    \label{fig:swrl_rule}
\end{figure}

\subsubsection{Regla 2: EncontrarMejorPrecio}

\begin{lstlisting}[style=sparql]
Producto(?p1) AND Producto(?p2) AND tieneNombre(?p1, ?n) AND tieneNombre(?p2, ?n) 
  AND tienePrecio(?p1, ?pr1) AND tienePrecio(?p2, ?pr2) AND lessThan(?pr1, ?pr2)
  -> esMejorOpcionQue(?p1, ?p2)
\end{lstlisting}

\textbf{Explicación:} Dos productos con el mismo nombre pero diferente precio; el de menor precio se marca como ``mejor opción que'' el otro.

\subsubsection{Regla 3: ClasificarPositivas}

\begin{lstlisting}[style=sparql]
Resena(?r) AND tieneCalificacion(?r, ?cal) AND greaterThanOrEqual(?cal, 4)
  -> Resena_Positiva(?r)
\end{lstlisting}

\subsubsection{Regla 4: ClasificarNegativas}

\begin{lstlisting}[style=sparql]
Resena(?r) AND tieneCalificacion(?r, ?cal) AND lessThanOrEqual(?cal, 2)
  -> Resena_Negativa(?r)
\end{lstlisting}

\subsection{Ejecución de Reglas}

El razonador Pellet se ejecuta al iniciar el backend:

\begin{lstlisting}[style=python]
# backend/reasoning/ontology_loader.py
from owlready2 import sync_reasoner_pellet

def load_ontology():
    onto = get_ontology("SmartCompareMarket.owl").load()
    
    # Ejecutar razonador Pellet con soporte SWRL
    with onto:
        sync_reasoner_pellet(infer_property_values=True, 
                             infer_data_property_values=True)
    
    return onto
\end{lstlisting}

\begin{figure}[H]
    \centering
    \includegraphics[width=0.9\textwidth]{logo_salle.png} 
    \caption{Log del backend mostrando Razonador Pellet ejecutado exitosamente}
    \label{fig:backend_log}
\end{figure}

\newpage

%%%%%%%%%%%%%%%%%%%%%%%%%%%%%%%%%%%%%%%%%%%%%%%%%%%%%%%%%%%%%%%%%%%%%%%%%%%%

\section{API REST - Endpoints}

\subsection{Base URL}

\begin{verbatim}
http://localhost:5000/api/v1
\end{verbatim}

\subsection{Endpoints de Productos}

\subsubsection{GET /products}

Obtener todos los productos.

\begin{lstlisting}
curl http://localhost:5000/api/v1/products
\end{lstlisting}

\textbf{Respuesta:}
\begin{lstlisting}[style=json]
{
  "products": [
    {
      "id": "Laptop_Dell_XPS",
      "name": "Dell XPS 15",
      "category": "Laptop",
      "types": ["Producto", "Electronica", "Computadora", "Laptop", "LaptopGamer"],
      "price": 1599.99,
      "ram_gb": 32,
      "storage_gb": 1024,
      "rating": 4.7
    }
  ],
  "count": 60
}
\end{lstlisting}

\begin{figure}[H]
    \centering
    \includegraphics[width=0.9\textwidth]{logo_salle.png} 
    \caption{Respuesta JSON de /api/v1/products en navegador}
    \label{fig:json_products}
\end{figure}

\subsubsection{GET /products/\{product\_id\}/relationships}

Obtener relaciones de un producto.

\begin{lstlisting}
curl http://localhost:5000/api/v1/products/iPhone15_Barato/relationships
\end{lstlisting}

\textbf{Respuesta:}
\begin{lstlisting}[style=json]
{
  "product_id": "iPhone15_Barato",
  "compatible": ["Cargador_USB_C", "Funda_iPhone15"],
  "incompatible": ["Cargador_MicroUSB"],
  "similar": ["iPhone15_Pro", "Samsung_Galaxy_S24"],
  "better_than": ["iPhone14_Base"]
}
\end{lstlisting}

\newpage

\subsection{Endpoint de Comparación}

\subsubsection{POST /compare}

\begin{lstlisting}
curl -X POST http://localhost:5000/api/v1/compare \
  -H "Content-Type: application/json" \
  -d '{"products": ["Laptop_Dell_XPS", "Laptop_MSI_Gaming"]}'
\end{lstlisting}

\textbf{Request Body:}
\begin{lstlisting}[style=json]
{
  "products": ["Laptop_Dell_XPS", "Laptop_MSI_Gaming", "Laptop_HP_Pavilion"]
}
\end{lstlisting}

\textbf{Respuesta:}
\begin{lstlisting}[style=json]
{
  "winner": "Laptop_Dell_XPS",
  "winner_score": 87.5,
  "comparison_table": {
    "Precio": {"Laptop_Dell_XPS": 1599, "Laptop_MSI_Gaming": 1299},
    "RAM (GB)": {"Laptop_Dell_XPS": 32, "Laptop_MSI_Gaming": 16},
    "Almacenamiento (GB)": {"Laptop_Dell_XPS": 1024, "Laptop_MSI_Gaming": 512}
  },
  "swrl_inferences": [
    {
      "type": "esMejorOpcionQue",
      "subject": "Laptop_Dell_XPS",
      "object": "Laptop_MSI_Gaming"
    }
  ],
  "reason": "Laptop_Dell_XPS gana con 87.5 puntos"
}
\end{lstlisting}

\begin{figure}[H]
    \centering
    \includegraphics[width=0.9\textwidth]{logo_salle.png} 
    \caption{Respuesta de comparación mostrando ganador y tabla}
    \label{fig:comparison_response}
\end{figure}

\subsection{Endpoint de Recomendaciones}

\subsubsection{POST /recommendations}

\begin{lstlisting}
curl -X POST http://localhost:5000/api/v1/recommendations \
  -H "Content-Type: application/json" \
  -d '{"budget": 1500, "preferred_category": "Laptop", "min_ram": 16}'
\end{lstlisting}

\textbf{Respuesta:}
\begin{lstlisting}[style=json]
{
  "recommendations": [
    {
      "product_id": "Laptop_MSI_Gaming",
      "name": "MSI GF65 Thin",
      "score": 92.5,
      "match_percentage": 95,
      "reason": "Laptop Gamer detectado (SWRL) + Excelente relacion calidad-precio",
      "price": 1299
    }
  ],
  "total_matches": 5
}
\end{lstlisting}

\begin{figure}[H]
    \centering
    \includegraphics[width=0.9\textwidth]{logo_salle.png} 
    \caption{Respuesta de recomendaciones con scores y razones}
    \label{fig:recommendation_response}
\end{figure}

\newpage

\subsection{Endpoint de Búsqueda SPARQL}

\subsubsection{GET /search}

\begin{lstlisting}
curl "http://localhost:5000/api/v1/search?category=Laptop&min_price=1000&max_price=1500&min_ram=16"
\end{lstlisting}

\textbf{Parámetros:}

\begin{table}[H]
	\centering
	\begin{tabular}{|l|l|p{6cm}|}
		\hline
		\rowcolor{tablebackground}
		\color{white}\textbf{Parámetro} & \color{white}\textbf{Tipo} & \color{white}\textbf{Descripción} \\
		\hline
		text & string & Búsqueda de texto libre \\
		\hline
		category & string & Categoría (Laptop, Smartphone, Tablet) \\
		\hline
		min\_price & float & Precio mínimo \\
		\hline
		max\_price & float & Precio máximo \\
		\hline
		min\_ram & integer & RAM mínima en GB \\
		\hline
	\end{tabular}
	\caption{Parámetros del endpoint de búsqueda}
\end{table}

\textbf{Consulta SPARQL generada internamente:}
\begin{lstlisting}[style=sparql]
PREFIX ns: <http://smartcompare.com/ontologia#>
PREFIX rdf: <http://www.w3.org/1999/02/22-rdf-syntax-ns#>

SELECT ?product ?name ?price ?ram WHERE {
  ?product rdf:type ns:Laptop .
  ?product ns:tienePrecio ?price .
  ?product ns:tieneRAM_GB ?ram .
  FILTER (?price >= 1000 && ?price <= 1500)
  FILTER (?ram >= 16)
}
\end{lstlisting}

\begin{figure}[H]
    \centering
    \includegraphics[width=0.9\textwidth]{logo_salle.png} 
    \caption{Respuesta de búsqueda filtrada}
    \label{fig:search_response}
\end{figure}

\subsection{Endpoints SWRL}

\subsubsection{GET /swrl/gaming-laptops}

\begin{lstlisting}
curl http://localhost:5000/api/v1/swrl/gaming-laptops
\end{lstlisting}

\textbf{Respuesta:}
\begin{lstlisting}[style=json]
{
  "rule": "DetectarGamer",
  "description": "Laptops con RAM >= 16GB clasificadas como LaptopGamer",
  "results": [
    {"id": "Laptop_Dell_XPS", "ram": 32, "classified_as": "LaptopGamer"},
    {"id": "Laptop_MSI_Gaming", "ram": 16, "classified_as": "LaptopGamer"}
  ],
  "count": 3
}
\end{lstlisting}

\begin{figure}[H]
    \centering
    \includegraphics[width=0.9\textwidth]{logo_salle.png} 
    \caption{Respuesta de gaming-laptops mostrando productos clasificados}
    \label{fig:swrl_response}
\end{figure}

\newpage

\subsection{Endpoint de Validación}

\subsubsection{GET /validate/all}

\begin{lstlisting}
curl http://localhost:5000/api/v1/validate/all
\end{lstlisting}

\textbf{Respuesta:}
\begin{lstlisting}[style=json]
{
  "summary": {
    "total_products": 60,
    "valid": 58,
    "with_errors": 1,
    "with_warnings": 3
  },
  "details": [...]
}
\end{lstlisting}

\begin{figure}[H]
    \centering
    \includegraphics[width=0.9\textwidth]{logo_salle.png} 
    \caption{Respuesta de validación masiva}
    \label{fig:validation_response}
\end{figure}

\subsection{Documentación Interactiva}

FastAPI genera documentación automática:

\begin{table}[H]
	\centering
	\begin{tabular}{|l|l|}
		\hline
		\rowcolor{tablebackground}
		\color{white}\textbf{URL} & \color{white}\textbf{Descripción} \\
		\hline
		http://localhost:5000/docs & Swagger UI (interactiva) \\
		\hline
		http://localhost:5000/redoc & ReDoc (documentación) \\
		\hline
	\end{tabular}
	\caption{URLs de documentación de la API}
\end{table}

\begin{figure}[H]
    \centering
    \includegraphics[width=0.9\textwidth]{logo_salle.png} 
    \caption{Página de Swagger UI en /docs mostrando todos los endpoints}
    \label{fig:swagger_ui}
\end{figure}

\newpage

%%%%%%%%%%%%%%%%%%%%%%%%%%%%%%%%%%%%%%%%%%%%%%%%%%%%%%%%%%%%%%%%%%%%%%%%%%%%

\section{Servicios del Backend}

\subsection{ComparisonService}

\textbf{Ubicación:} \texttt{backend/services/comparison\_service.py} (445 líneas)

\textbf{Responsabilidad:} Motor de comparación inteligente con scoring multi-factor.

\subsubsection{Algoritmo de Scoring}

\begin{lstlisting}[style=python]
WEIGHTS = {
    "battery": 0.20,      # Mayor es mejor
    "rating": 0.18,       # Mayor es mejor
    "price": 0.14,        # MENOR es mejor (invertido)
    "resolution": 0.10,   # Mayor es mejor
    "ram": 0.10,          # Mayor es mejor
    "storage": 0.10,      # Mayor es mejor
    "warranty": 0.07,     # Mayor es mejor
    "screen": 0.06,       # Mayor es mejor
    "weight": 0.05        # MENOR es mejor (invertido)
}

REFERENCE_VALUES = {
    "battery": 10000,     # 10000 mAh = 100 puntos
    "rating": 5.0,        # 5 estrellas = 100 puntos
    "price": 3000,        # $3000 = 0 puntos (precio maximo)
    "ram": 64,            # 64GB = 100 puntos
    "storage": 2048,      # 2TB = 100 puntos
}
\end{lstlisting}

\subsubsection{Cálculo del Score}

\begin{lstlisting}[style=python]
def _calculate_score(self, product: dict) -> float:
    score = 0
    
    for factor, weight in WEIGHTS.items():
        value = product.get(factor, 0)
        reference = REFERENCE_VALUES[factor]
        
        if factor in ["price", "weight"]:  # Menor es mejor
            normalized = max(0, 100 - (value / reference * 100))
        else:  # Mayor es mejor
            normalized = min(100, value / reference * 100)
        
        score += normalized * weight
    
    # Bonus por reglas SWRL
    if self._has_swrl_bonus(product):
        score += 2  # Bonus por cada relacion "esMejorOpcionQue"
    
    return round(score, 2)
\end{lstlisting}

\begin{figure}[H]
    \centering
    \includegraphics[width=0.9\textwidth]{logo_salle.png} 
    \caption{Código de _calculate_score en comparison_service.py}
    \label{fig:calc_score}
\end{figure}

\newpage

\subsection{RecommendationService}

\textbf{Ubicación:} \texttt{backend/services/recommendation\_service.py} (277 líneas)

\textbf{Responsabilidad:} Sistema de recomendaciones personalizadas.

\subsubsection{Sistema de Scoring para Recomendaciones}

\begin{lstlisting}[style=python]
def _calculate_recommendation_score(self, product: dict, preferences: dict) -> float:
    score = 0
    
    # Factor 1: Presupuesto (30 puntos)
    if product["price"] <= preferences["budget"]:
        budget_usage = product["price"] / preferences["budget"]
        score += 30 * budget_usage
    
    # Factor 2: Calificacion (25 puntos)
    score += product["rating"] * 5  # 5 puntos por estrella
    
    # Factor 3: RAM (15 puntos)
    if product["ram"] >= preferences.get("min_ram", 0):
        score += 15
    
    # Factor 4: Almacenamiento (10 puntos)
    if product["storage"] >= preferences.get("min_storage", 0):
        score += 10
    
    # Bonus SWRL
    if "LaptopGamer" in product.get("types", []):
        score += 10  # Bonus por ser Laptop Gamer
    
    return score
\end{lstlisting}

\subsection{ValidationService}

\textbf{Ubicación:} \texttt{backend/services/validation\_service.py} (152 líneas)

\textbf{Reglas de Validación:}

\begin{table}[H]
	\centering
	\begin{tabular}{|l|p{8cm}|}
		\hline
		\rowcolor{tablebackground}
		\color{white}\textbf{Tipo} & \color{white}\textbf{Condición} \\
		\hline
		Error & Precio negativo \\
		\hline
		Error & RAM negativa o $>$ 512GB \\
		\hline
		Error & Almacenamiento negativo o $>$ 10TB \\
		\hline
		Error & Calificación fuera de rango 0-5 \\
		\hline
		Advertencia & Precio $>$ \$100,000 \\
		\hline
		Advertencia & RAM $>$ 128GB \\
		\hline
	\end{tabular}
	\caption{Reglas de validación implementadas}
\end{table}

\newpage

%%%%%%%%%%%%%%%%%%%%%%%%%%%%%%%%%%%%%%%%%%%%%%%%%%%%%%%%%%%%%%%%%%%%%%%%%%%%

\section{Instalación para Desarrolladores}

\subsection{Prerrequisitos}

\begin{table}[H]
	\centering
	\begin{tabular}{|l|l|l|}
		\hline
		\rowcolor{tablebackground}
		\color{white}\textbf{Software} & \color{white}\textbf{Versión} & \color{white}\textbf{Verificación} \\
		\hline
		Python & 3.11+ & \texttt{python --version} \\
		\hline
		Node.js & 18+ & \texttt{node --version} \\
		\hline
		npm & 9+ & \texttt{npm --version} \\
		\hline
		Java JDK & 11+ & \texttt{java -version} \\
		\hline
		Git & Cualquiera & \texttt{git --version} \\
		\hline
	\end{tabular}
	\caption{Prerrequisitos de instalación}
\end{table}

\begin{figure}[H]
    \centering
    \includegraphics[width=0.9\textwidth]{logo_salle.png} 
    \caption{Terminal mostrando todos los comandos de verificación}
    \label{fig:verify_cmds}
\end{figure}

\subsection{Clonar el Repositorio}

\begin{lstlisting}
git clone https://github.com/AlvaroMachaca0503/Web_Semantica.git
cd Web_Semantica
\end{lstlisting}

\subsection{Configuración del Backend}

\begin{lstlisting}
# 1. Navegar al backend
cd backend

# 2. Crear entorno virtual
python -m venv venv

# 3. Activar entorno virtual
# Windows:
venv\Scripts\activate
# Linux/Mac:
source venv/bin/activate

# 4. Instalar dependencias
pip install -r requirements.txt

# 5. Verificar instalacion
python -c "import owlready2; print('Owlready2 OK')"
python -c "import fastapi; print('FastAPI OK')"
\end{lstlisting}

\begin{figure}[H]
    \centering
    \includegraphics[width=0.9\textwidth]{logo_salle.png} 
    \caption{Terminal con entorno virtual activado y verificaciones exitosas}
    \label{fig:venv_verify}
\end{figure}

\subsection{Configuración del Frontend}

\begin{lstlisting}
# 1. Navegar al frontend
cd frontend

# 2. Instalar dependencias
npm install

# 3. Verificar instalacion
npm list react
\end{lstlisting}

\newpage

%%%%%%%%%%%%%%%%%%%%%%%%%%%%%%%%%%%%%%%%%%%%%%%%%%%%%%%%%%%%%%%%%%%%%%%%%%%%

\section{Ejecución y Despliegue}

\subsection{Modo Desarrollo}

\textbf{Terminal 1 - Backend:}
\begin{lstlisting}
cd backend
venv\Scripts\activate  # Windows
python main.py
\end{lstlisting}

\textbf{Logs esperados:}
\begin{lstlisting}
[INFO] Cargando ontologia SmartCompareMarket.owl...
[OK] Ontologia cargada: 60 productos encontrados
[INFO] Ejecutando razonador Pellet...
[OK] Razonador Pellet ejecutado exitosamente
[OK] Reglas SWRL aplicadas: DetectarGamer, EncontrarMejorPrecio...
INFO:     Uvicorn running on http://0.0.0.0:5000
\end{lstlisting}

\begin{figure}[H]
    \centering
    \includegraphics[width=0.9\textwidth]{logo_salle.png} 
    \caption{Terminal del backend con todos los logs de inicio exitosos}
    \label{fig:backend_start}
\end{figure}

\textbf{Terminal 2 - Frontend:}
\begin{lstlisting}
cd frontend
npm run dev
\end{lstlisting}

\textbf{Logs esperados:}
\begin{lstlisting}
  VITE v5.4.19  ready in 500 ms

  ->  Local:   http://localhost:5173/
\end{lstlisting}

\begin{figure}[H]
    \centering
    \includegraphics[width=0.9\textwidth]{logo_salle.png} 
    \caption{Terminal del frontend con Vite ejecutándose}
    \label{fig:frontend_start}
\end{figure}

\subsection{Modo Producción}

\textbf{Backend:}
\begin{lstlisting}
cd backend
uvicorn main:app --host 0.0.0.0 --port 5000 --workers 4
\end{lstlisting}

\textbf{Frontend:}
\begin{lstlisting}
cd frontend
npm run build
npm run preview
\end{lstlisting}

\subsection{Puertos Utilizados}

\begin{table}[H]
	\centering
	\begin{tabular}{|l|l|l|}
		\hline
		\rowcolor{tablebackground}
		\color{white}\textbf{Servicio} & \color{white}\textbf{Puerto} & \color{white}\textbf{URL} \\
		\hline
		Backend API & 5000 & http://localhost:5000 \\
		\hline
		Frontend Dev & 5173 & http://localhost:5173 \\
		\hline
		Swagger Docs & 5000 & http://localhost:5000/docs \\
		\hline
	\end{tabular}
	\caption{Puertos utilizados por el sistema}
\end{table}

\newpage

%%%%%%%%%%%%%%%%%%%%%%%%%%%%%%%%%%%%%%%%%%%%%%%%%%%%%%%%%%%%%%%%%%%%%%%%%%%%

\section{Testing}

\subsection{Tests del Backend}

\begin{lstlisting}
cd backend

# Ejecutar todos los tests
pytest

# Con cobertura
pytest --cov=.

# Tests especificos
pytest tests/test_comparison.py -v
\end{lstlisting}

\subsection{Tests de la API}

\begin{lstlisting}
# Test basico de salud
curl http://localhost:5000/api/v1/products | head -c 200

# Test de comparacion
curl -X POST http://localhost:5000/api/v1/compare \
  -H "Content-Type: application/json" \
  -d '{"products": ["Laptop_Dell_XPS", "Laptop_MSI_Gaming"]}'
\end{lstlisting}

\subsection{Tests Manuales Recomendados}

\begin{table}[H]
	\centering
	\begin{tabular}{|l|p{4cm}|p{5cm}|}
		\hline
		\rowcolor{tablebackground}
		\color{white}\textbf{Test} & \color{white}\textbf{Comando} & \color{white}\textbf{Resultado Esperado} \\
		\hline
		Listar productos & GET /api/v1/products & JSON con 60+ productos \\
		\hline
		Gaming laptops & GET /api/v1/swrl/gaming-laptops & 3+ laptops con RAM $\geq$ 16GB \\
		\hline
		Comparación & POST /api/v1/compare & Ganador con score \\
		\hline
		Validación & GET /api/v1/validate/all & Resumen de validación \\
		\hline
	\end{tabular}
	\caption{Tests manuales recomendados}
\end{table}

\begin{figure}[H]
    \centering
    \includegraphics[width=0.9\textwidth]{logo_salle.png} 
    \caption{Resultados de pytest mostrando todos los tests pasando}
    \label{fig:pytest_results}
\end{figure}

\newpage

%%%%%%%%%%%%%%%%%%%%%%%%%%%%%%%%%%%%%%%%%%%%%%%%%%%%%%%%%%%%%%%%%%%%%%%%%%%%

\section{Mantenimiento y Extensibilidad}

\subsection{Agregar Nuevos Productos}

\begin{enumerate}
	\item Abrir \texttt{backend/ontology/SmartCompareMarket.owl} en \textbf{Protégé}
	\item Tab ``Individuals'' $\rightarrow$ Click ``+'' para agregar individuo
	\item Seleccionar clase (ej: \texttt{Laptop})
	\item Agregar propiedades de datos:
	\begin{itemize}
		\item \texttt{tienePrecio}
		\item \texttt{tieneRAM\_GB}
		\item \texttt{tieneAlmacenamiento\_GB}
	\end{itemize}
	\item Guardar archivo
	\item Reiniciar backend
\end{enumerate}

\begin{figure}[H]
    \centering
    \includegraphics[width=0.9\textwidth]{logo_salle.png} 
    \caption{Protégé mostrando cómo agregar un nuevo individuo}
    \label{fig:protege_add}
\end{figure}

\subsection{Agregar Nuevas Reglas SWRL}

\begin{enumerate}
	\item En Protégé $\rightarrow$ Tab ``SWRL''
	\item Click ``+'' para nueva regla
	\item Escribir regla en formato SWRL
	\item Guardar
	\item Reiniciar backend
\end{enumerate}

\textbf{Ejemplo de nueva regla:}
\begin{lstlisting}[style=sparql]
Smartphone(?s) AND tieneRAM_GB(?s, ?ram) AND greaterThan(?ram, 8)
  -> SmartphoneGamaAlta(?s)
\end{lstlisting}

\subsection{Modificar Pesos de Comparación}

Editar archivo \texttt{backend/data/comparison\_weights.json}:

\begin{lstlisting}[style=json]
{
  "battery": 0.20,
  "rating": 0.18,
  "price": 0.14,
  "resolution": 0.10,
  "ram": 0.10,
  "storage": 0.10,
  "warranty": 0.07,
  "screen": 0.06,
  "weight": 0.05
}
\end{lstlisting}

\newpage

%%%%%%%%%%%%%%%%%%%%%%%%%%%%%%%%%%%%%%%%%%%%%%%%%%%%%%%%%%%%%%%%%%%%%%%%%%%%

\section{Lista de Capturas de Pantalla}

\begin{table}[H]
	\centering
	\begin{tabular}{|c|p{11cm}|}
		\hline
		\rowcolor{tablebackground}
		\color{white}\textbf{\#} & \color{white}\textbf{Descripción de la Captura} \\
		\hline
		1 & Diagrama de arquitectura del sistema \\
		\hline
		2 & Archivo \texttt{requirements.txt} con dependencias \\
		\hline
		3 & Archivo \texttt{package.json} con dependencias \\
		\hline
		4 & Estructura de carpetas del proyecto \\
		\hline
		5 & Jerarquía de clases en Protégé \\
		\hline
		6 & Propiedades de datos en Protégé \\
		\hline
		7 & Propiedades de objeto en Protégé \\
		\hline
		8 & Individuos en Protégé \\
		\hline
		9 & Regla DetectarGamer en Protégé \\
		\hline
		10 & Log del backend con Razonador Pellet \\
		\hline
		11 & Respuesta JSON de /api/v1/products \\
		\hline
		12 & Respuesta de comparación \\
		\hline
		13 & Respuesta de recomendaciones \\
		\hline
		14 & Respuesta de búsqueda SPARQL \\
		\hline
		15 & Respuesta de gaming-laptops \\
		\hline
		16 & Respuesta de validación masiva \\
		\hline
		17 & Swagger UI en /docs \\
		\hline
		18 & Código de \_calculate\_score \\
		\hline
		19 & Terminal con comandos de verificación \\
		\hline
		20 & Terminal con entorno virtual activado \\
		\hline
		21 & Terminal del backend ejecutándose \\
		\hline
		22 & Terminal del frontend ejecutándose \\
		\hline
		23 & Resultados de pytest \\
		\hline
		24 & Protégé agregando nuevo individuo \\
		\hline
	\end{tabular}
	\caption{Lista de capturas de pantalla requeridas}
\end{table}

\textbf{Total de capturas requeridas: 24}

\newpage

%%%%%%%%%%%%%%%%%%%%%%%%%%%%%%%%%%%%%%%%%%%%%%%%%%%%%%%%%%%%%%%%%%%%%%%%%%%%

\section{Resumen de Archivos Importantes}

\begin{table}[H]
	\centering
	\begin{tabular}{|l|l|p{5cm}|}
		\hline
		\rowcolor{tablebackground}
		\color{white}\textbf{Archivo} & \color{white}\textbf{Líneas} & \color{white}\textbf{Descripción} \\
		\hline
		SmartCompareMarket.owl & 2,900+ & Ontología principal \\
		\hline
		comparison\_service.py & 445 & Motor de comparación \\
		\hline
		recommendation\_service.py & 277 & Sistema de recomendaciones \\
		\hline
		queries.py & 267 & Consultas SPARQL \\
		\hline
		validation\_service.py & 152 & Validación de datos \\
		\hline
		main.py & $\sim$200 & Punto de entrada \\
		\hline
		ComparePage.tsx & 500+ & Página de comparación \\
		\hline
	\end{tabular}
	\caption{Archivos principales del proyecto}
\end{table}

\newpage

%%%%%%%%%%%%%%%%%%%%%%%%%%%%%%%%%%%%%%%%%%%%%%%%%%%%%%%%%%%%%%%%%%%%%%%%%%%%

\section{Información del Proyecto}

\begin{table}[H]
	\centering
	\begin{tabular}{|p{5cm}|p{9cm}|}
		\hline
		\rowcolor{tablebackground}
		\color{white}\textbf{Campo} & \color{white}\textbf{Información} \\
		\hline
		Universidad & \itemUniversity \\
		\hline
		Facultad & \itemFaculty \\
		\hline
		Carrera & \itemSchool \\
		\hline
		Curso & \itemCourse \\
		\hline
		Docente & \itemProfesor \\
		\hline
		Ciclo Académico & \itemAcademic \\
		\hline
		Fecha & \itemDate \\
		\hline
		Proyecto & \itemProjectNumber \\
		\hline
		Nombre del Sistema & \itemProjectName \\
		\hline
		Versión del Manual & \itemVersion \\
		\hline
	\end{tabular}
	\caption{Información académica del proyecto}
\end{table}

\subsection{Autor}

\begin{table}[H]
	\centering
	\begin{tabular}{|l|l|}
		\hline
		\rowcolor{tablebackground}
		\color{white}\textbf{Nombre} & \color{white}\textbf{Correo Electrónico} \\
		\hline
		Álvaro André Machaca Meléndez & amachacam@ulasalle.edu.pe \\
		\hline
	\end{tabular}
	\caption{Autor del proyecto}
\end{table}

\subsection{Repositorio}

\begin{itemize}
	\item \textbf{GitHub:} \url{https://github.com/AlvaroMachaca0503/Web_Semantica}
\end{itemize}

\subsection{Historial de Versiones}

\begin{table}[H]
	\centering
	\begin{tabular}{|c|c|p{8cm}|}
		\hline
		\rowcolor{tablebackground}
		\color{white}\textbf{Versión} & \color{white}\textbf{Fecha} & \color{white}\textbf{Cambios} \\
		\hline
		1.0 & Diciembre 2024 & Versión inicial del manual técnico \\
		\hline
	\end{tabular}
	\caption{Control de versiones del manual}
\end{table}

\end{document}
